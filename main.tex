\documentclass[12pt]{article}

\usepackage{graphicx}
\usepackage{amsmath}
\usepackage{hyperref}

\title{The Domestic Cat: A Study of Behavior, Characteristics, and Importance}
\author{Izaan Aamir}
\date{\today}

\begin{document}

\maketitle

\begin{abstract}
The domestic cat (\textit{Felis catus}) is one of the world’s most popular and beloved animals. 
This paper provides an overview of cats, including their physical characteristics, behavior, 
diet, communication, and their role in human society. Cats have lived alongside humans for thousands 
of years and continue to be valued for companionship and their natural hunting abilities.
\end{abstract}

\section{Introduction}
Cats are small, carnivorous mammals that have been domesticated for nearly 10,000 years. 
Known for their agility, independence, and affectionate behavior, cats are now one of the 
most common household pets around the world. This paper discusses various aspects of cats, 
including their biology, behavior, and importance to humans.

\section{Physical Characteristics}
Cats have highly developed senses that make them excellent hunters and companions.
Some key physical features include:
\begin{itemize}
    \item Sharp retractable claws
    \item Excellent night vision
    \item Strong sense of smell and hearing
    \item Flexible spine and powerful muscles
\end{itemize}
These traits make cats agile, fast, and capable of jumping great heights.

\section{Behavior and Communication}
Cats communicate using a mixture of:
\begin{itemize}
    \item Vocalizations such as meowing, purring, and hissing
    \item Body language including tail movement and ear positioning
    \item Scent marking to establish territory
\end{itemize}
Cats are known for being independent, but they also form strong bonds with their owners and 
can be very affectionate.

\section{Diet}
Cats are obligate carnivores, meaning they require meat to survive. Their diet generally 
consists of:
\begin{itemize}
    \item Small animals such as mice and birds
    \item High-protein commercial cat food
    \item Occasional treats like fish or chicken
\end{itemize}
A healthy diet is essential for cat growth and long-term wellbeing.

\section{Role in Human Society}
Cats have been valued throughout history for several reasons:
\begin{itemize}
    \item Companionship and emotional support
    \item Natural rodent control in homes and farms
    \item Cultural significance in art, mythology, and literature
\end{itemize}

\section{Conclusion}
Cats are fascinating animals with unique physical and behavioral traits. Their relationship 
with humans continues to evolve, but one thing remains constant: cats are cherished members 
of many households and play an important role in ecosystems and human life.

\section*{References}
\begin{thebibliography}{9}

\bibitem{ref1}
National Geographic. \textit{Cat Facts and Information}. 2024.

\bibitem{ref2}
American Veterinary Association. \textit{Cat Behavior and Care}. 2023.

\end{thebibliography}

\end{document}
